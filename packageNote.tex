\documentclass{article}\usepackage[]{graphicx}\usepackage[]{color}
%% maxwidth is the original width if it is less than linewidth
%% otherwise use linewidth (to make sure the graphics do not exceed the margin)
\makeatletter
\def\maxwidth{ %
  \ifdim\Gin@nat@width>\linewidth
    \linewidth
  \else
    \Gin@nat@width
  \fi
}
\makeatother

\definecolor{fgcolor}{rgb}{0.345, 0.345, 0.345}
\newcommand{\hlnum}[1]{\textcolor[rgb]{0.686,0.059,0.569}{#1}}%
\newcommand{\hlstr}[1]{\textcolor[rgb]{0.192,0.494,0.8}{#1}}%
\newcommand{\hlcom}[1]{\textcolor[rgb]{0.678,0.584,0.686}{\textit{#1}}}%
\newcommand{\hlopt}[1]{\textcolor[rgb]{0,0,0}{#1}}%
\newcommand{\hlstd}[1]{\textcolor[rgb]{0.345,0.345,0.345}{#1}}%
\newcommand{\hlkwa}[1]{\textcolor[rgb]{0.161,0.373,0.58}{\textbf{#1}}}%
\newcommand{\hlkwb}[1]{\textcolor[rgb]{0.69,0.353,0.396}{#1}}%
\newcommand{\hlkwc}[1]{\textcolor[rgb]{0.333,0.667,0.333}{#1}}%
\newcommand{\hlkwd}[1]{\textcolor[rgb]{0.737,0.353,0.396}{\textbf{#1}}}%

\usepackage{framed}
\makeatletter
\newenvironment{kframe}{%
 \def\at@end@of@kframe{}%
 \ifinner\ifhmode%
  \def\at@end@of@kframe{\end{minipage}}%
  \begin{minipage}{\columnwidth}%
 \fi\fi%
 \def\FrameCommand##1{\hskip\@totalleftmargin \hskip-\fboxsep
 \colorbox{shadecolor}{##1}\hskip-\fboxsep
     % There is no \\@totalrightmargin, so:
     \hskip-\linewidth \hskip-\@totalleftmargin \hskip\columnwidth}%
 \MakeFramed {\advance\hsize-\width
   \@totalleftmargin\z@ \linewidth\hsize
   \@setminipage}}%
 {\par\unskip\endMakeFramed%
 \at@end@of@kframe}
\makeatother

\definecolor{shadecolor}{rgb}{.97, .97, .97}
\definecolor{messagecolor}{rgb}{0, 0, 0}
\definecolor{warningcolor}{rgb}{1, 0, 1}
\definecolor{errorcolor}{rgb}{1, 0, 0}
\newenvironment{knitrout}{}{} % an empty environment to be redefined in TeX

\usepackage{alltt}
\usepackage{CJKutf8}
%\DeclareUnicodeCharacter{0020}{~}
%\setCJKmainfont{KaiTi}
\usepackage{amsfonts}
\usepackage{amsmath}
\title{R Packages Note}
\author{Appsoner}
\IfFileExists{upquote.sty}{\usepackage{upquote}}{}
\begin{document}
\begin{CJK*}{UTF8}{gkai}
\maketitle
\tableofcontents
%\section{R packages}
\section{ggplot2}
\begin{knitrout}
\definecolor{shadecolor}{rgb}{0.969, 0.969, 0.969}\color{fgcolor}\begin{kframe}
\begin{alltt}
\hlkwd{library}\hlstd{(ggplot2)}
\end{alltt}
\end{kframe}
\end{knitrout}
\subsection{qplot}
\begin{knitrout}
\definecolor{shadecolor}{rgb}{0.969, 0.969, 0.969}\color{fgcolor}\begin{kframe}
\begin{alltt}
\hlkwd{set.seed}\hlstd{(}\hlnum{1410}\hlstd{)}
\hlstd{dsmall} \hlkwb{<-} \hlstd{diamonds[}\hlkwd{sample}\hlstd{(}\hlkwd{nrow}\hlstd{(diamonds),} \hlnum{100}\hlstd{), ]}
\end{alltt}
\end{kframe}
\end{knitrout}
尝试\textbf{qplot}
\begin{knitrout}
\definecolor{shadecolor}{rgb}{0.969, 0.969, 0.969}\color{fgcolor}\begin{kframe}
\begin{alltt}
\hlstd{opts_chunk}\hlopt{$}\hlkwd{set}\hlstd{(}\hlkwc{fig.height} \hlstd{=} \hlnum{5}\hlstd{)}
\hlkwd{qplot}\hlstd{(carat, price,} \hlkwc{data} \hlstd{= diamonds)}
\end{alltt}
\end{kframe}
\includegraphics[width=\maxwidth]{figure/firstPlot1} 
\begin{kframe}\begin{alltt}
\hlkwd{qplot}\hlstd{(}\hlkwd{log}\hlstd{(carat),} \hlkwd{log}\hlstd{(price),} \hlkwc{data} \hlstd{= diamonds)}
\end{alltt}
\end{kframe}
\includegraphics[width=\maxwidth]{figure/firstPlot2} 
\begin{kframe}\begin{alltt}
\hlkwd{qplot}\hlstd{(carat, x}\hlopt{*}\hlstd{y}\hlopt{*}\hlstd{z,} \hlkwc{data} \hlstd{= diamonds)}
\end{alltt}
\end{kframe}
\includegraphics[width=\maxwidth]{figure/firstPlot3} 
\begin{kframe}\begin{alltt}
\hlkwd{qplot}\hlstd{(carat, price,} \hlkwc{data} \hlstd{= dsmall,} \hlkwc{colour} \hlstd{= color)}
\end{alltt}
\end{kframe}
\includegraphics[width=\maxwidth]{figure/firstPlot4} 
\begin{kframe}\begin{alltt}
\hlkwd{qplot}\hlstd{(carat, price,} \hlkwc{data} \hlstd{= dsmall,} \hlkwc{shape} \hlstd{= cut)}
\end{alltt}
\end{kframe}
\includegraphics[width=\maxwidth]{figure/firstPlot5} 

\end{knitrout}
通过\textbf{I()}手动设定图形属性,使用透明属性\textbf{alpha},~$1/10$~重叠~$10$~后变得不透明。
\begin{knitrout}
\definecolor{shadecolor}{rgb}{0.969, 0.969, 0.969}\color{fgcolor}\begin{kframe}
\begin{alltt}
\hlkwd{qplot}\hlstd{(carat, price,} \hlkwc{data} \hlstd{= diamonds,} \hlkwc{alpha} \hlstd{=} \hlkwd{I}\hlstd{(}\hlnum{1}\hlopt{/}\hlnum{10}\hlstd{))}
\end{alltt}
\end{kframe}
\includegraphics[width=\maxwidth]{figure/I___alpha1} 
\begin{kframe}\begin{alltt}
\hlkwd{qplot}\hlstd{(carat, price,} \hlkwc{data} \hlstd{= diamonds,} \hlkwc{alpha} \hlstd{=} \hlkwd{I}\hlstd{(}\hlnum{1}\hlopt{/}\hlnum{100}\hlstd{))}
\end{alltt}
\end{kframe}
\includegraphics[width=\maxwidth]{figure/I___alpha2} 
\begin{kframe}\begin{alltt}
\hlkwd{qplot}\hlstd{(carat, price,} \hlkwc{data} \hlstd{= diamonds,} \hlkwc{alpha} \hlstd{=} \hlkwd{I}\hlstd{(}\hlnum{1}\hlopt{/}\hlnum{200}\hlstd{))}
\end{alltt}
\end{kframe}
\includegraphics[width=\maxwidth]{figure/I___alpha3} 

\end{knitrout}
添加平滑曲线,通过参数\em{se}决定是否需要划出置信区间。
\begin{knitrout}
\definecolor{shadecolor}{rgb}{0.969, 0.969, 0.969}\color{fgcolor}\begin{kframe}
\begin{alltt}
\hlkwd{qplot}\hlstd{(carat, price,} \hlkwc{data} \hlstd{= dsmall,} \hlkwc{geom} \hlstd{=} \hlkwd{c}\hlstd{(}\hlstr{"point"}\hlstd{,} \hlstr{"smooth"}\hlstd{))}
\end{alltt}
\end{kframe}
\includegraphics[width=\maxwidth]{figure/add_line1} 
\begin{kframe}\begin{alltt}
\hlkwd{qplot}\hlstd{(carat, price,} \hlkwc{data} \hlstd{= diamonds,} \hlkwc{geom} \hlstd{=} \hlkwd{c}\hlstd{(}\hlstr{"point"}\hlstd{,} \hlstr{"smooth"}\hlstd{))}
\end{alltt}
\end{kframe}
\includegraphics[width=\maxwidth]{figure/add_line2} 
\begin{kframe}\begin{alltt}
\hlkwd{qplot}\hlstd{(carat, price,} \hlkwc{data} \hlstd{= dsmall,} \hlkwc{geom} \hlstd{=} \hlkwd{c}\hlstd{(}\hlstr{"point"}\hlstd{,} \hlstr{"smooth"}\hlstd{),} \hlkwc{se} \hlstd{= F)}
\end{alltt}
\end{kframe}
\includegraphics[width=\maxwidth]{figure/add_line3} 

\end{knitrout}
\em{method}决定平滑方法,\em{span}控制平滑程度,~$O$~(很不平滑)到~$1$~(很平滑)
\begin{knitrout}
\definecolor{shadecolor}{rgb}{0.969, 0.969, 0.969}\color{fgcolor}\begin{kframe}
\begin{alltt}
\hlkwd{qplot}\hlstd{(carat, price,} \hlkwc{data} \hlstd{= dsmall,} \hlkwc{geom} \hlstd{=} \hlkwd{c}\hlstd{(}\hlstr{"point"}\hlstd{,} \hlstr{"smooth"}\hlstd{),} \hlkwc{span} \hlstd{=} \hlnum{0.2}\hlstd{)}
\end{alltt}


{\ttfamily\noindent\itshape\color{messagecolor}{\#\# geom\_smooth: method="{}auto"{} and size of largest group is <1000, so using loess. Use 'method = x' to change the smoothing method.}}\end{kframe}
\includegraphics[width=\maxwidth]{figure/span1} 
\begin{kframe}\begin{alltt}
\hlkwd{qplot}\hlstd{(carat, price,} \hlkwc{data} \hlstd{= dsmall,} \hlkwc{geom} \hlstd{=} \hlkwd{c}\hlstd{(}\hlstr{"point"}\hlstd{,} \hlstr{"smooth"}\hlstd{),} \hlkwc{span} \hlstd{=} \hlnum{1}\hlstd{)}
\end{alltt}


{\ttfamily\noindent\itshape\color{messagecolor}{\#\# geom\_smooth: method="{}auto"{} and size of largest group is <1000, so using loess. Use 'method = x' to change the smoothing method.}}\end{kframe}
\includegraphics[width=\maxwidth]{figure/span2} 

\end{knitrout}
通过mgcv包拟合一个广义可加模型。对于大数据,超过~$1000$~默认使用~$y~s(x, bs = "cs")$~。此外,还有$lm$和$rlm$,后者更加稳健,对异常值不敏感。
\begin{knitrout}
\definecolor{shadecolor}{rgb}{0.969, 0.969, 0.969}\color{fgcolor}\begin{kframe}
\begin{alltt}
\hlkwd{library}\hlstd{(mgcv)}
\hlkwd{library}\hlstd{(MASS)}
\hlkwd{qplot}\hlstd{(carat, price,} \hlkwc{data} \hlstd{= dsmall,} \hlkwc{geom} \hlstd{=} \hlkwd{c}\hlstd{(}\hlstr{"point"}\hlstd{,} \hlstr{"smooth"}\hlstd{),}
      \hlkwc{method} \hlstd{=} \hlstr{"gam"}\hlstd{,} \hlkwc{formula} \hlstd{= y} \hlopt{~} \hlkwd{s}\hlstd{(x))}
\end{alltt}
\end{kframe}
\includegraphics[width=\maxwidth]{figure/method1} 
\begin{kframe}\begin{alltt}
\hlkwd{qplot}\hlstd{(carat, price,} \hlkwc{data} \hlstd{= dsmall,} \hlkwc{geom} \hlstd{=} \hlkwd{c}\hlstd{(}\hlstr{"point"}\hlstd{,} \hlstr{"smooth"}\hlstd{),}
      \hlkwc{method} \hlstd{=} \hlstr{"gam"}\hlstd{,} \hlkwc{formula} \hlstd{= y} \hlopt{~} \hlkwd{s}\hlstd{(x,} \hlkwc{bs} \hlstd{=} \hlstr{"cs"}\hlstd{))}
\end{alltt}
\end{kframe}
\includegraphics[width=\maxwidth]{figure/method2} 
\begin{kframe}\begin{alltt}
\hlkwd{qplot}\hlstd{(carat, price,} \hlkwc{data} \hlstd{= dsmall,} \hlkwc{geom} \hlstd{=} \hlkwd{c}\hlstd{(}\hlstr{"point"}\hlstd{,} \hlstr{"smooth"}\hlstd{),}
      \hlkwc{method} \hlstd{=} \hlstr{"lm"}\hlstd{)}
\end{alltt}
\end{kframe}
\includegraphics[width=\maxwidth]{figure/method3} 
\begin{kframe}\begin{alltt}
\hlkwd{qplot}\hlstd{(carat, price,} \hlkwc{data} \hlstd{= dsmall,} \hlkwc{geom} \hlstd{=} \hlkwd{c}\hlstd{(}\hlstr{"point"}\hlstd{,} \hlstr{"smooth"}\hlstd{),}
      \hlkwc{method} \hlstd{=} \hlstr{"rlm"}\hlstd{)}
\end{alltt}
\end{kframe}
\includegraphics[width=\maxwidth]{figure/method4} 

\end{knitrout}
观察连续变量随分类变量的水平变化。通过箱形图和扰定点图来观察。
\begin{knitrout}
\definecolor{shadecolor}{rgb}{0.969, 0.969, 0.969}\color{fgcolor}\begin{kframe}
\begin{alltt}
\hlkwd{qplot}\hlstd{(color, price}\hlopt{/}\hlstd{carat,} \hlkwc{data} \hlstd{= diamonds,} \hlkwc{geom} \hlstd{=} \hlkwd{c}\hlstd{(}\hlstr{"jitter"}\hlstd{))}
\end{alltt}
\end{kframe}
\includegraphics[width=\maxwidth]{figure/category1} 
\begin{kframe}\begin{alltt}
\hlkwd{qplot}\hlstd{(color, price}\hlopt{/}\hlstd{carat,} \hlkwc{data} \hlstd{= diamonds,} \hlkwc{geom} \hlstd{=} \hlkwd{c}\hlstd{(}\hlstr{"boxplot"}\hlstd{))}
\end{alltt}
\end{kframe}
\includegraphics[width=\maxwidth]{figure/category2} 

\end{knitrout}
为了看清出数据的集中范围,我们可以通过调节透明度来调节图形重叠。
\begin{knitrout}
\definecolor{shadecolor}{rgb}{0.969, 0.969, 0.969}\color{fgcolor}\begin{kframe}
\begin{alltt}
\hlkwd{qplot}\hlstd{(color, price}\hlopt{/}\hlstd{carat,} \hlkwc{data} \hlstd{= diamonds,} \hlkwc{geom} \hlstd{=} \hlkwd{c}\hlstd{(}\hlstr{"jitter"}\hlstd{),} \hlkwc{alpha} \hlstd{=} \hlkwd{I}\hlstd{(}\hlnum{1}\hlopt{/}\hlnum{5}\hlstd{))}
\end{alltt}
\end{kframe}
\includegraphics[width=\maxwidth]{figure/ad_overlap1} 
\begin{kframe}\begin{alltt}
\hlkwd{qplot}\hlstd{(color, price}\hlopt{/}\hlstd{carat,} \hlkwc{data} \hlstd{= diamonds,} \hlkwc{geom} \hlstd{=} \hlkwd{c}\hlstd{(}\hlstr{"jitter"}\hlstd{),} \hlkwc{alpha} \hlstd{=} \hlkwd{I}\hlstd{(}\hlnum{1}\hlopt{/}\hlnum{50}\hlstd{))}
\end{alltt}
\end{kframe}
\includegraphics[width=\maxwidth]{figure/ad_overlap2} 
\begin{kframe}\begin{alltt}
\hlkwd{qplot}\hlstd{(color, price}\hlopt{/}\hlstd{carat,} \hlkwc{data} \hlstd{= diamonds,} \hlkwc{geom} \hlstd{=} \hlkwd{c}\hlstd{(}\hlstr{"jitter"}\hlstd{),} \hlkwc{alpha} \hlstd{=} \hlkwd{I}\hlstd{(}\hlnum{1}\hlopt{/}\hlnum{200}\hlstd{))}
\end{alltt}
\end{kframe}
\includegraphics[width=\maxwidth]{figure/ad_overlap3} 

\end{knitrout}
直方图和密度曲线图。
\begin{knitrout}
\definecolor{shadecolor}{rgb}{0.969, 0.969, 0.969}\color{fgcolor}\begin{kframe}
\begin{alltt}
\hlkwd{qplot}\hlstd{(carat,} \hlkwc{data} \hlstd{= diamonds,} \hlkwc{geom} \hlstd{=} \hlstr{"histogram"}\hlstd{)}
\end{alltt}


{\ttfamily\noindent\itshape\color{messagecolor}{\#\# stat\_bin: binwidth defaulted to range/30. Use 'binwidth = x' to adjust this.}}\end{kframe}
\includegraphics[width=\maxwidth]{figure/his_density1} 
\begin{kframe}\begin{alltt}
\hlkwd{qplot}\hlstd{(carat,} \hlkwc{data} \hlstd{= diamonds,} \hlkwc{geom} \hlstd{=} \hlstr{"density"}\hlstd{)}
\end{alltt}
\end{kframe}
\includegraphics[width=\maxwidth]{figure/his_density2} 

\end{knitrout}
调节平滑程度是十分重要的,尝试多种组距;组距较大,显示总体特征那;组距小,突出细节。组距通过\em{binwidth}来调节。
\begin{knitrout}
\definecolor{shadecolor}{rgb}{0.969, 0.969, 0.969}\color{fgcolor}\begin{kframe}
\begin{alltt}
\hlkwd{qplot}\hlstd{(carat,} \hlkwc{data} \hlstd{= diamonds,} \hlkwc{geom} \hlstd{=} \hlstr{"histogram"}\hlstd{,} \hlkwc{binwidth} \hlstd{=} \hlnum{1}\hlstd{,}
      \hlkwc{xlim} \hlstd{=} \hlkwd{c}\hlstd{(}\hlnum{0}\hlstd{,} \hlnum{3}\hlstd{))}
\end{alltt}
\end{kframe}
\includegraphics[width=\maxwidth]{figure/binwidth1} 
\begin{kframe}\begin{alltt}
\hlkwd{qplot}\hlstd{(carat,} \hlkwc{data} \hlstd{= diamonds,} \hlkwc{geom} \hlstd{=} \hlstr{"histogram"}\hlstd{,} \hlkwc{binwidth} \hlstd{=} \hlnum{0.1}\hlstd{,}
      \hlkwc{xlim} \hlstd{=} \hlkwd{c}\hlstd{(}\hlnum{0}\hlstd{,} \hlnum{3}\hlstd{))}
\end{alltt}
\end{kframe}
\includegraphics[width=\maxwidth]{figure/binwidth2} 
\begin{kframe}\begin{alltt}
\hlkwd{qplot}\hlstd{(carat,} \hlkwc{data} \hlstd{= diamonds,} \hlkwc{geom} \hlstd{=} \hlstr{"histogram"}\hlstd{,} \hlkwc{binwidth} \hlstd{=} \hlnum{0.01}\hlstd{,}
      \hlkwc{xlim} \hlstd{=} \hlkwd{c}\hlstd{(}\hlnum{0}\hlstd{,} \hlnum{3}\hlstd{))}
\end{alltt}


{\ttfamily\noindent\color{warningcolor}{\#\# Warning: position\_stack requires constant width: output may be incorrect}}\end{kframe}
\includegraphics[width=\maxwidth]{figure/binwidth3} 

\end{knitrout}
在不同组之间进行比较,加上一个类别变量,也叫做\textbf{图形映射}
\begin{knitrout}
\definecolor{shadecolor}{rgb}{0.969, 0.969, 0.969}\color{fgcolor}\begin{kframe}
\begin{alltt}
\hlkwd{qplot}\hlstd{(carat,} \hlkwc{data} \hlstd{= diamonds,} \hlkwc{geom} \hlstd{=} \hlstr{"histogram"}\hlstd{,} \hlkwc{fill} \hlstd{= color)}
\end{alltt}


{\ttfamily\noindent\itshape\color{messagecolor}{\#\# stat\_bin: binwidth defaulted to range/30. Use 'binwidth = x' to adjust this.}}\end{kframe}
\includegraphics[width=\maxwidth]{figure/shape_projection1} 
\begin{kframe}\begin{alltt}
\hlkwd{qplot}\hlstd{(carat,} \hlkwc{data} \hlstd{= diamonds,} \hlkwc{geom} \hlstd{=} \hlstr{"density"}\hlstd{,} \hlkwc{colour} \hlstd{= color)}
\end{alltt}
\end{kframe}
\includegraphics[width=\maxwidth]{figure/shape_projection2} 

\end{knitrout}
条形图,\em{weight}对连续性变量进行加权处理(如对连续型变量进行分组求和。
\begin{knitrout}
\definecolor{shadecolor}{rgb}{0.969, 0.969, 0.969}\color{fgcolor}\begin{kframe}
\begin{alltt}
\hlkwd{qplot}\hlstd{(color,} \hlkwc{data} \hlstd{= diamonds,} \hlkwc{geom} \hlstd{=} \hlstr{"bar"}\hlstd{)}
\end{alltt}
\end{kframe}
\includegraphics[width=\maxwidth]{figure/bar1} 
\begin{kframe}\begin{alltt}
\hlkwd{qplot}\hlstd{(color,} \hlkwc{data} \hlstd{= diamonds,} \hlkwc{geom} \hlstd{=} \hlstr{"bar"}\hlstd{,} \hlkwc{weight} \hlstd{= carat)} \hlopt{+}
\hlkwd{scale_y_continuous}\hlstd{(}\hlstr{"carat"}\hlstd{)}
\end{alltt}
\end{kframe}
\includegraphics[width=\maxwidth]{figure/bar2} 

\end{knitrout}






\section{shiny}
\section{quantmod}
\section{quantmod}
\subsection{获得数据}
\subsubsection{getSymbols}
\begin{knitrout}
\definecolor{shadecolor}{rgb}{0.969, 0.969, 0.969}\color{fgcolor}\begin{kframe}
\begin{alltt}
\hlstd{opts_chunk}\hlopt{$}\hlkwd{set}\hlstd{(}\hlkwc{fig.height} \hlstd{=} \hlnum{5}\hlstd{)}
\hlkwd{library}\hlstd{(quantmod)}
\end{alltt}
\end{kframe}
\end{knitrout}
设定环境,环境是一个个对象的容器,创建一个新的环境是在当前工作环境中开辟一个新的环境空间,用来存储想单独存储的变量或对象等。
\begin{knitrout}
\definecolor{shadecolor}{rgb}{0.969, 0.969, 0.969}\color{fgcolor}\begin{kframe}
\begin{alltt}
\hlstd{new.environment} \hlkwb{<-} \hlkwd{new.env}\hlstd{()} \hlcom{#新建环境空间,存储对象}
\hlkwd{getSymbols}\hlstd{(}\hlstr{"IBM"}\hlstd{,} \hlkwc{env} \hlstd{= new.environment,} \hlkwc{src} \hlstd{=} \hlstr{'yahoo'}\hlstd{,}
           \hlkwc{from} \hlstd{=} \hlstr{"2013-10-01"}\hlstd{,} \hlkwc{to} \hlstd{=} \hlstr{"2013-10-23"}\hlstd{)}
\end{alltt}
\begin{verbatim}
## [1] "IBM"
\end{verbatim}
\end{kframe}
\end{knitrout}
\begin{knitrout}
\definecolor{shadecolor}{rgb}{0.969, 0.969, 0.969}\color{fgcolor}\begin{kframe}
\begin{alltt}
\hlkwd{get}\hlstd{(}\hlstr{"IBM"}\hlstd{,} \hlkwc{env} \hlstd{= new.environment)}
\end{alltt}
\begin{verbatim}
##            IBM.Open IBM.High IBM.Low IBM.Close IBM.Volume IBM.Adjusted
## 2013-10-01    185.3    186.7   184.7     186.4    2681200        182.2
## 2013-10-02    185.5    186.3   184.4     185.0    3617100        180.8
## 2013-10-03    184.7    185.0   183.0     183.9    3211800        179.8
## 2013-10-04    184.2    185.1   183.6     184.1    2863600        180.0
## 2013-10-07    181.8    183.3   181.8     182.0    3966400        178.0
## 2013-10-08    181.9    182.0   178.7     178.7    5578300        174.7
## 2013-10-09    179.4    181.7   179.1     181.3    4423500        177.3
## 2013-10-10    183.2    184.8   182.4     184.8    3658900        180.7
## 2013-10-11    185.2    186.2   184.1     186.2    3232600        182.0
## 2013-10-14    185.4    187.0   184.4     187.0    2663100        182.8
## 2013-10-15    185.7    185.9   184.2     184.7    3365100        180.6
## 2013-10-16    185.4    186.7   185.0     186.7    6718000        182.6
## 2013-10-17    173.8    177.0   172.6     174.8   22368900        170.9
## 2013-10-18    174.8    175.0   173.2     173.8   10548000        169.9
## 2013-10-21    174.4    174.8   172.6     172.9    7098700        169.0
## 2013-10-22    173.3    175.6   172.9     175.0    6977300        171.1
## 2013-10-23    175.1    176.0   174.4     175.8    5409400        171.9
\end{verbatim}
\end{kframe}
\end{knitrout}
\subsubsection{setSymbolLookup}
也可以通过~\textbf{setSymbolLookup}~进行设置
\begin{knitrout}
\definecolor{shadecolor}{rgb}{0.969, 0.969, 0.969}\color{fgcolor}\begin{kframe}
\begin{alltt}
\hlkwd{setSymbolLookup}\hlstd{(}\hlkwc{IBM} \hlstd{=} \hlkwd{list}\hlstd{(}\hlkwc{name} \hlstd{=} \hlstr{'IBM'}\hlstd{,} \hlkwc{src} \hlstd{=} \hlstr{'yahoo'}\hlstd{),}
                \hlkwc{USDEUR} \hlstd{=} \hlkwd{list}\hlstd{(}\hlkwc{name} \hlstd{=} \hlstr{'USD/EUR'}\hlstd{,} \hlkwc{src} \hlstd{=} \hlstr{'oanda'}\hlstd{))}
\hlkwd{getSymbols}\hlstd{(}\hlkwd{c}\hlstd{(}\hlstr{'IBM'}\hlstd{,} \hlstr{'USDEUR'}\hlstd{))}
\end{alltt}
\begin{verbatim}
## [1] "IBM"    "USDEUR"
\end{verbatim}
\begin{alltt}
\hlkwd{head}\hlstd{(IBM)}
\end{alltt}
\begin{verbatim}
##            IBM.Open IBM.High IBM.Low IBM.Close IBM.Volume IBM.Adjusted
## 2007-01-03    97.18    98.40   96.26     97.27    9196800        84.57
## 2007-01-04    97.25    98.79   96.88     98.31   10524500        85.48
## 2007-01-05    97.60    97.95   96.91     97.42    7221300        84.70
## 2007-01-08    98.50    99.50   98.35     98.90   10340000        85.99
## 2007-01-09    99.08   100.33   99.07    100.07   11108200        87.01
## 2007-01-10    98.50    99.05   97.93     98.89    8744800        85.98
\end{verbatim}
\end{kframe}
\end{knitrout}
\subsubsection{getFX}
获取汇率(从oanda上)
\begin{knitrout}
\definecolor{shadecolor}{rgb}{0.969, 0.969, 0.969}\color{fgcolor}\begin{kframe}
\begin{alltt}
\hlkwd{getFX}\hlstd{(}\hlstr{"HKD/USD"}\hlstd{,} \hlkwc{from} \hlstd{=} \hlstr{"2013-10-20"}\hlstd{,} \hlkwc{to} \hlstd{=} \hlstr{"2013-10-25"}\hlstd{,}
      \hlkwc{env} \hlstd{= new.environment)}
\end{alltt}
\begin{verbatim}
## [1] "HKDUSD"
\end{verbatim}
\begin{alltt}
\hlkwd{get}\hlstd{(}\hlstr{'HKDUSD'}\hlstd{,} \hlkwc{env} \hlstd{= new.environment)}
\end{alltt}
\begin{verbatim}
##            HKD.USD
## 2013-10-20   0.129
## 2013-10-21   0.129
## 2013-10-22   0.129
## 2013-10-23   0.129
## 2013-10-24   0.129
## 2013-10-25   0.129
\end{verbatim}
\end{kframe}
\end{knitrout}
\subsubsection{getFinancials}
\begin{knitrout}
\definecolor{shadecolor}{rgb}{0.969, 0.969, 0.969}\color{fgcolor}\begin{kframe}
\begin{alltt}
\hlkwd{getFinancials}\hlstd{(}\hlstr{"APPL"}\hlstd{)}
\hlkwd{viewFinancials}\hlstd{(APPL.f)}
\end{alltt}
\end{kframe}
\end{knitrout}
\subsubsection{getDividends}
\em{verbose = T}显示抓取的过程。
\begin{knitrout}
\definecolor{shadecolor}{rgb}{0.969, 0.969, 0.969}\color{fgcolor}\begin{kframe}
\begin{alltt}
\hlkwd{getDividends}\hlstd{(}\hlstr{"IBM"}\hlstd{,} \hlkwc{from} \hlstd{=} \hlstr{"2012-01-01"}\hlstd{,}
             \hlkwc{to} \hlstd{=} \hlstr{"2013-10-25"}\hlstd{,} \hlkwc{verbose} \hlstd{= T)}
\end{alltt}
\end{kframe}
\end{knitrout}
\subsection{is,has族函数}
\em{O}开盘价,\em{H}最高价,\em{L}最低价,\em{C}收盘价,\em{V}成交量
\begin{knitrout}
\definecolor{shadecolor}{rgb}{0.969, 0.969, 0.969}\color{fgcolor}\begin{kframe}
\begin{alltt}
\hlkwd{is.OHLC}\hlstd{(IBM)}
\end{alltt}
\begin{verbatim}
## [1] TRUE
\end{verbatim}
\begin{alltt}
\hlkwd{is.OHLCV}\hlstd{(IBM)}
\end{alltt}
\begin{verbatim}
## [1] TRUE
\end{verbatim}
\begin{alltt}
\hlkwd{is.BBO}\hlstd{(IBM)}
\end{alltt}
\begin{verbatim}
## [1] FALSE
\end{verbatim}
\begin{alltt}
\hlkwd{is.TBBO}\hlstd{(IBM)}
\end{alltt}
\begin{verbatim}
## [1] FALSE
\end{verbatim}
\begin{alltt}
\hlkwd{is.HLC}\hlstd{(IBM)}
\end{alltt}
\begin{verbatim}
## [1] TRUE
\end{verbatim}
\end{kframe}
\end{knitrout}
\begin{knitrout}
\definecolor{shadecolor}{rgb}{0.969, 0.969, 0.969}\color{fgcolor}\begin{kframe}
\begin{alltt}
\hlkwd{has.OHLC}\hlstd{(IBM)}
\end{alltt}
\begin{verbatim}
## [1] TRUE TRUE TRUE TRUE
\end{verbatim}
\begin{alltt}
\hlkwd{has.OHLC}\hlstd{(IBM,} \hlkwc{which} \hlstd{=} \hlnum{FALSE}\hlstd{)}
\end{alltt}
\begin{verbatim}
## [1] TRUE TRUE TRUE TRUE
\end{verbatim}
\begin{alltt}
\hlkwd{has.OHLC}\hlstd{(IBM,} \hlkwc{which} \hlstd{=} \hlnum{TRUE}\hlstd{)}
\end{alltt}
\begin{verbatim}
## [1] 1 2 3 4
\end{verbatim}
\end{kframe}
\end{knitrout}
\subsection{列名函数数据}
\subsubsection{OHLC,Op}
\begin{knitrout}
\definecolor{shadecolor}{rgb}{0.969, 0.969, 0.969}\color{fgcolor}\begin{kframe}
\begin{alltt}
\hlkwd{head}\hlstd{(}\hlkwd{OHLC}\hlstd{(IBM))}
\end{alltt}
\begin{verbatim}
##            IBM.Open IBM.High IBM.Low IBM.Close
## 2007-01-03    97.18    98.40   96.26     97.27
## 2007-01-04    97.25    98.79   96.88     98.31
## 2007-01-05    97.60    97.95   96.91     97.42
## 2007-01-08    98.50    99.50   98.35     98.90
## 2007-01-09    99.08   100.33   99.07    100.07
## 2007-01-10    98.50    99.05   97.93     98.89
\end{verbatim}
\begin{alltt}
\hlkwd{head}\hlstd{(}\hlkwd{Op}\hlstd{(IBM))}
\end{alltt}
\begin{verbatim}
##            IBM.Open
## 2007-01-03    97.18
## 2007-01-04    97.25
## 2007-01-05    97.60
## 2007-01-08    98.50
## 2007-01-09    99.08
## 2007-01-10    98.50
\end{verbatim}
\end{kframe}
\end{knitrout}
\subsection{计算函数}
\subsubsection{Delt}
主要是用来计算一个序列的一个阶段到另一个阶段的变化率或者计算两个序列之间的变化率
\begin{knitrout}
\definecolor{shadecolor}{rgb}{0.969, 0.969, 0.969}\color{fgcolor}\begin{kframe}
\begin{alltt}
\hlkwd{head}\hlstd{(}\hlkwd{Delt}\hlstd{(}\hlkwd{Op}\hlstd{(IBM),} \hlkwc{type} \hlstd{=} \hlstr{"arithmetic"}\hlstd{))} \hlcom{#lag 1期}
\end{alltt}
\begin{verbatim}
##            Delt.1.arithmetic
## 2007-01-03                NA
## 2007-01-04         0.0007203
## 2007-01-05         0.0035990
## 2007-01-08         0.0092213
## 2007-01-09         0.0058883
## 2007-01-10        -0.0058539
\end{verbatim}
\begin{alltt}
\hlkwd{head}\hlstd{(}\hlkwd{Delt}\hlstd{(}\hlkwd{Op}\hlstd{(IBM),} \hlkwc{type} \hlstd{= (}\hlstr{"log"}\hlstd{)))}
\end{alltt}
\begin{verbatim}
##            Delt.1.log
## 2007-01-03         NA
## 2007-01-04  0.0007201
## 2007-01-05  0.0035925
## 2007-01-08  0.0091791
## 2007-01-09  0.0058711
## 2007-01-10 -0.0058711
\end{verbatim}
\begin{alltt}
\hlkwd{head}\hlstd{(}\hlkwd{Delt}\hlstd{(}\hlkwd{Op}\hlstd{(IBM),} \hlkwd{Cl}\hlstd{(IBM)))}
\end{alltt}
\begin{verbatim}
##            Delt.0.arithmetic
## 2007-01-03         0.0009261
## 2007-01-04         0.0108997
## 2007-01-05        -0.0018443
## 2007-01-08         0.0040609
## 2007-01-09         0.0099919
## 2007-01-10         0.0039594
\end{verbatim}
\end{kframe}
\end{knitrout}
\subsubsection{first,lasth函数}
\begin{knitrout}
\definecolor{shadecolor}{rgb}{0.969, 0.969, 0.969}\color{fgcolor}\begin{kframe}
\begin{alltt}
\hlkwd{first}\hlstd{(IBM)}
\end{alltt}
\begin{verbatim}
##            IBM.Open IBM.High IBM.Low IBM.Close IBM.Volume IBM.Adjusted
## 2007-01-03    97.18     98.4   96.26     97.27    9196800        84.57
\end{verbatim}
\begin{alltt}
\hlkwd{last}\hlstd{(IBM)}
\end{alltt}
\begin{verbatim}
##            IBM.Open IBM.High IBM.Low IBM.Close IBM.Volume IBM.Adjusted
## 2014-08-29    192.3    192.8   191.1     192.3    2909400        192.3
\end{verbatim}
\end{kframe}
\end{knitrout}
\subsubsection{Next函数}
\begin{knitrout}
\definecolor{shadecolor}{rgb}{0.969, 0.969, 0.969}\color{fgcolor}\begin{kframe}
\begin{alltt}
\hlkwd{head}\hlstd{(IBM)}
\end{alltt}
\begin{verbatim}
##            IBM.Open IBM.High IBM.Low IBM.Close IBM.Volume IBM.Adjusted
## 2007-01-03    97.18    98.40   96.26     97.27    9196800        84.57
## 2007-01-04    97.25    98.79   96.88     98.31   10524500        85.48
## 2007-01-05    97.60    97.95   96.91     97.42    7221300        84.70
## 2007-01-08    98.50    99.50   98.35     98.90   10340000        85.99
## 2007-01-09    99.08   100.33   99.07    100.07   11108200        87.01
## 2007-01-10    98.50    99.05   97.93     98.89    8744800        85.98
\end{verbatim}
\begin{alltt}
\hlkwd{head}\hlstd{(}\hlkwd{Next}\hlstd{(IBM,}\hlnum{1}\hlstd{))}
\end{alltt}
\begin{verbatim}
##             Next
## 2007-01-03 97.25
## 2007-01-04 97.60
## 2007-01-05 98.50
## 2007-01-08 99.08
## 2007-01-09 98.50
## 2007-01-10 99.00
\end{verbatim}
\end{kframe}
\end{knitrout}
\subsubsection{to.weekly和to.monthly函数}
开盘价为一周第一天的开盘价,收盘价为一周最后一天周五的收盘价。最高价为一周中的最高价,最低价为一周中的最低价。成交量为一周总的成交量。
汇总月的类比。
\begin{knitrout}
\definecolor{shadecolor}{rgb}{0.969, 0.969, 0.969}\color{fgcolor}\begin{kframe}
\begin{alltt}
\hlkwd{head}\hlstd{(}\hlkwd{to.weekly}\hlstd{(IBM))}
\end{alltt}
\begin{verbatim}
##            IBM.Open IBM.High IBM.Low IBM.Close IBM.Volume IBM.Adjusted
## 2007-01-05    97.18    98.79   96.26     97.42   26942600        84.70
## 2007-01-12    98.50   100.33   97.93     99.34   44830200        86.37
## 2007-01-19    99.40   100.90   94.55     96.17   58474800        83.62
## 2007-01-26    96.42    97.92   96.12     97.45   41549100        84.73
## 2007-02-02    97.70    99.73   97.45     99.17   34172300        86.22
## 2007-02-09    99.17   100.44   97.81     98.55   34668400        85.94
\end{verbatim}
\begin{alltt}
\hlkwd{head}\hlstd{(}\hlkwd{to.monthly}\hlstd{(IBM))}
\end{alltt}


{\ttfamily\noindent\color{warningcolor}{\#\# Warning: timezone of object (UTC) is different than current timezone ().}}\begin{verbatim}
##           IBM.Open IBM.High IBM.Low IBM.Close IBM.Volume IBM.Adjusted
##  1月 2007    97.18   100.90   94.55     99.15  192702000        86.21
##  2月 2007    98.97   100.44   92.47     92.94  125776200        81.05
##  3月 2007    90.25    95.81   88.77     94.26  163194500        82.20
##  4月 2007    94.51   103.00   93.91    102.21  166807200        89.13
##  5月 2007   102.06   108.05  101.35    106.60  146441400        93.32
##  6月 2007   106.62   107.24  101.56    105.25  169602000        92.14
\end{verbatim}
\end{kframe}
\end{knitrout}

\newpage
\end{CJK*}
\end{document}
