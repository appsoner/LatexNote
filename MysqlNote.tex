\documentclass[UTF8]{ctexart}
\setCJKmainfont{KaiTi}
\usepackage{amsfonts}
\usepackage{amsmath}
\usepackage{listings}
\usepackage{xcolor}
\lstset{
%alsolanguage=Java, 
alsolanguage=SQL, 
%language={[ISO]C++},       %language为,还有{[Visual]C++}  
%alsolanguage=[ANSI]C,      %可以添加很多个alsolanguage,如alsolanguage=matlab,alsolanguage=VHDL等  
%alsolanguage= tcl,  
alsolanguage= XML,  
tabsize=4, %  
frame=shadowbox, %把代码用带有阴影的框圈起来  
commentstyle=\color{red!50!green!50!blue!50},%浅灰色的注释  
rulesepcolor=\color{red!20!green!20!blue!20},%代码块边框为淡青色  
keywordstyle=\color{blue!90}\bfseries, %代码关键字的颜色为蓝色,粗体  
showstringspaces=false,%不显示代码字符串中间的空格标记  
stringstyle=\ttfamily, % 代码字符串的特殊格式  
keepspaces=true, %  
breakindent=22pt, %  
numbers=left,%左侧显示行号 往左靠,还可以为right,或none,即不加行号  
stepnumber=1,%若设置为2,则显示行号为1,3,5,即stepnumber为公差,默认stepnumber=1  
%numberstyle=\tiny, %行号字体用小号  
numberstyle={\color[RGB]{0,192,192}\tiny} ,%设置行号的大小,大小有tiny,scriptsize,footnotesize,small,normalsize,large等  
numbersep=8pt,  %设置行号与代码的距离,默认是5pt  
basicstyle=\footnotesize, % 这句设置代码的大小  
showspaces=false, %  
flexiblecolumns=true, %  
breaklines=true, %对过长的代码自动换行  
breakautoindent=true,%  
breakindent=4em, %  
%escapebegin=\begin{CJK*}{GBK}{hei},escapeend=\end{CJK*},  
aboveskip=1em, %代码块边框  
tabsize=2,  
showstringspaces=false, %不显示字符串中的空格  
backgroundcolor=\color[RGB]{245,245,244},   %代码背景色  
%backgroundcolor=\color[rgb]{0.91,0.91,0.91}    %添加背景色  
escapeinside=``,  %在``里显示中文  
%% added by http://bbs.ctex.org/viewthread.php?tid=53451  
fontadjust,  
captionpos=t,  
framextopmargin=2pt,framexbottommargin=2pt,abovecaptionskip=-3pt,belowcaptionskip=3pt,  
xleftmargin=4em,xrightmargin=4em, % 设定listing左右的空白  
texcl=true,% 设定中文冲突,断行,列模式,数学环境输入,listing数字的样式  
%extendedchars=false,columns=flexible,mathescape=true  
%numbersep=-1em  
}  
\title{MySql笔记}
\author{徐国盛}
\begin{document}
\maketitle
\tableofcontents
\section{数据和表}
1.CEATE DATABASE 创建数据库

如:CREATE DATABASE my$\_$contacts

2.USE DATABASE(name) 进入相应的数据库

如:USE DATABASE my$\_$contacts

3.CREATE TABLE (table name) (column1, column2,\dots)  创建表

如:CREATE TABLE my$\_$contacts (

			last$\_$name VARCHAR(30) NOT NULL,

			firs$\_$name VARCHAR(20) NOT NULL,

			birthday DATA NOT NULL,

			\dots

			);

CREATE TABLE doughnut$\_$list (

			 doughnut$\_$name VARCHAR(20) NOT NULL,
 
			 doughnut$\_$type VARCHAR(6) NOT NULL,

			 doughnut$\_$cost DEC(3,2) NOT NULL DEFAULT 1.00

 );

4.INSERT INTO tablename (column1, column2, \dots) VALUES(data1, data2, \dots) 插入数据

5.DROP TABLE tablename 删除表

要点:

1.如果想查看表的结构,可以使用DESC。

2.DROP TABLE 语句用于丢弃表,慎重使用!

3.为表插入数据时,可以使用任何一种INSERT语句。

4.NULL是未定义的值。它不等于零,也不等于空值。值可以是NULL,但觉非等于NULL。

5.没有在INSERT语句中赋值的列默认为NULL。

6.可以把列修改为不接受NULL值,这需要创建表时使用关键字NOT NULL。

7.创建表时使用DEFAULT,可于日后输入缺乏部分数据的记录时自动填入默认值。
\section{SELECT语句}
1.SELECT * FROM tablename 选出此表中所有的行列。

如:SELECT * FROM my\_contacts

2.MySql中,'用\textbackslash'或’’表示。
3.SELECT column1 column2 \dots FROM tablename WHERE (the conditions)
	
如:SELECT drink\_name FROM easy\_drinks WHERE second = 'lemonade'。其中可以有$>$,$<$,$=$和$<>$等运算符号。

4.SELECT column FROM tablename WHERE condition1  AND condition2;

如:SELECT drink\_name FROM easy\_drinks WHERE main = 'soda' AND amount1 = 1.5;其中可以运用OR等关系关键字。	
	
SELECT drink\_name FROM easy\_drinks WHERE	cost >= 3.5 AND calories < 50;

SELECT drink\_name FROM drink\_info WHERE drink\_name >= 'L' AND drink\_name < 'M'; 选出名字手字母为L的的饮料。	

5.找到NULL的列

1>直接找到NULL

SELECT drink\_name FROM drink\_info WHERE calories IS NULL;

2>间接找的NULL

SELECT calories FROM drink\_info WHERE drink\_name = 'Dragon Breath'; 在这里,名为‘Dragon Breath’的饮料calories为NULL。通过确定其他列来寻找calories值为NULL的行

6.通配符

SELECT * FROM my\_contacts WHERE location LIKE '$\%$CA';其中,\%是一个通配符,可以等于字符串中的任何字符。在这里是找出以地址CA结尾的行。

SELECT first\_name FROM mu\_contacts WHERE first\_name LIKE '\_IM';下划线\_只是一个字符的替身。

7.确定一个范围。

1>用AND和比较运算符确定一个范围。

SELECT drink\_name FROM drink\_info WHERE calories >= 30 AND calories <= 60;

2>用BETWEEN确定一个范围。

SELECT drink\_name FROM drink\_info WHERE calories BETWEEN 30 AND 60;

8. IN和NOT IN

1>SELECT date\_name FROM black\_book WHERE rating IN ('innovative', 'fabulous', 'delightful', 'pretty good');

2>SELECT date\_name FROM black\_book WHERE rating NOT IN ('innovative', 'fabulous', 'delightful', 'pretty good');

\textbf{注意}:NOT 可以和BETWEEN或LIKE一起使用。\textbf{NOT 一定要紧跟在WHERE后面,当NOT和AND或OR一起使用时,要直接接在AND或OR后面。}如下:

SELECT drink\_name FROM drink\_info WHERE NOT carbs BETWEEN 3 AND 5;

SELECT date\_name FROM black\_book WHERE NOT date\_name LIKE 'A\%' AND NOT date\_name LIKE 'B\%';

但是会有以下之中特殊情况:

1> NOT IN

SELECT * FROM easy\_drinks WHERE NOT main IN ('soda', 'iced tea');

SELECT * FROM easy\_drinks WHERE main NOT IN ('soda', 'iced tea'); 效果是一样的。

2> NULL 取得列中所有不是NULL的值

SELECT * FROM easy\_drinks WHERE NOT main IS NULL;

SELECT * FROM easy\_drinks WHERE mian IS NOT NULL; 效果是一样的。
\section{DELETE和UPDATE}
DELETE FROM clown\_info WHERE activities = 'dancing'; 删除where判断确定的列

\textbf{DELETE同样可以和关系运算符AND,0R,$<>$计算,但要谨慎使用。一般情况下先通过SELECT语句选出要删除的行,进行确定,然后在通过DELETE语句删除。}

UPDATE doughnut\_ratings SET type = 'glazed' WHERE type = 'plain glazed';

UPDATE doughnut\_ratings SET cost = cost + 1 WHERE cost = 3;

UPDATE语句时更新WHERE语句确定的行的其中的列值,使用时是十分安全的;在某些情况下,UPDATE = SELECT + DELETE。同时,在SET项中,可以进行相关的数值运算和字符运算。
\section{聪明的表设计}

ATOMIC DATA 数据原子性。列中的数据若以拆解成查询所需的最小单位,就具有原子性。

ATOMIC DATA 规则一:

具有原子性表示在同一列中不会存储多个类型相同的数据。

ATOMIC DATA 规则二:

具有原子性表示不会用多个列来存储类型相同的数据。

FIRST NORMAL FORM(1NR) 第一范式。每个数据行均包含原子性数据值。而且每个数据行均需具有唯一的识别方法。

1> SHOW CREATE TABLE 使用这个命令来呈现创建现有表的正确语法。如:
\begin{lstlisting}[language = SQL]
SHOW CREATE TABLE my\_mycontacts
\end{lstlisting}

2> PRIMARY KEY ()设置主键。一个或一组能识别出唯一数据行的列。

3> AUTO\_INCREMENT 在列的声明中使用这个关键字,每次执行INSERT命令来插入数据时,它都会自动给列赋予唯一的递增数据值。如:
\begin{lstlisting}[language = SQL]
CREATE TABLE persons (
			id INT NOT NULL AUTO\_INCREMENT,
			first\_name VARCHAR(30) NOT NULL,
			last\_name VARCHAR(30) NOT NULL,
			PRIMARY KEY (id) );
\end{lstlisting}
3> ALTER 语句。如:
\begin{lstlisting}[language = SQL]
ALTER TABLE my\_contacts 
ADD COLUMN contact\_id INT NOT NULL AUTO\_INCREMENT FIRST,
ADD PRIMARY KEY (contact\_id);
\end{lstlisting}
\section{ALTER}
\section{SELECT进阶}
\section{多张表的数据库设计}
\section{连接与多张表的操作}
\section{子查询}
\section{外联接、自联接与联合}
\section{约束、视图与事务}
\section{安全性}
\section{附录1}
\section{附录2}
\section{附录3}
\end{document}
