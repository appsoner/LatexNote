\documentclass[UTF8]{ctexart}
%\usepackage{CJKutf8}
\setCJKmainfont{KaiTi}
\usepackage{graphicx}
\usepackage{amsfonts}
\usepackage{amsmath}
\usepackage{listings}
\usepackage{xcolor}
\lstset{%
alsolanguage = Python,	
%alsolanguage=Java,  
%language={[ISO]C++},       %language为,还有{[Visual]C++}  
%alsolanguage=[ANSI]C,      %可以添加很多个alsolanguage,如alsolanguage=matlab,alsolanguage=VHDL等  
%alsolanguage= tcl,  
alsolanguage= XML,  
tabsize=4, %  
frame=shadowbox, %把代码用带有阴影的框圈起来  
commentstyle=\color{red!60!blue!90},%浅灰色的注释  
rulesepcolor=\color{red!20!green!20!blue!20},%代码块边框为淡青色  
keywordstyle=\color{blue!90}\bfseries, %代码关键字的颜色为蓝色,粗体  
showstringspaces=false,%不显示代码字符串中间的空格标记  
stringstyle=\ttfamily, % 代码字符串的特殊格式  
keepspaces=true, %  
breakindent=22pt, %  
numbers=left,%左侧显示行号 往左靠,还可以为right,或none,即不加行号  
stepnumber=1,%若设置为2,则显示行号为1,3,5,即stepnumber为公差,默认stepnumber=1  
%numberstyle=\tiny, %行号字体用小号  
numberstyle={\color[RGB]{0,192,192}\tiny} ,%设置行号的大小,大小有tiny,scriptsize,footnotesize,small,normalsize,large等  
numbersep=8pt,  %设置行号与代码的距离,默认是5pt  
basicstyle=\footnotesize, % 这句设置代码的大小  
showspaces=false, %  
flexiblecolumns=true, %  
breaklines=true, %对过长的代码自动换行  
breakautoindent=true,%  
breakindent=4em, %  
%escapebegin=\begin{CJK*}{GBK}{hei},escapeend=\end{CJK*},  
aboveskip=1em, %代码块边框  
tabsize=2,  
showstringspaces=false, %不显示字符串中的空格  
backgroundcolor=\color[RGB]{245,245,244},   %代码背景色  
%backgroundcolor=\color[rgb]{0.91,0.91,0.91}    %添加背景色  
escapeinside=``,  %在``里显示中文  
%% added by http://bbs.ctex.org/viewthread.php?tid=53451  
fontadjust,  
captionpos=t,  
framextopmargin=2pt,framexbottommargin=2pt,abovecaptionskip=-3pt,belowcaptionskip=3pt,  
xleftmargin=4em,xrightmargin=4em, % 设定listing左右的空白  
texcl=true,% 设定中文冲突,断行,列模式,数学环境输入,listing数字的样式  
%extendedchars=false,columns=flexible,mathescape=true  
%numbersep=-1em  
}  
\title{Git Note}
\author{徐国盛}
\begin{document}
\maketitle
\tableofcontents
\section{仓库初始化}

\textbullet 初始化一个仓库,使用\textit{git init}命令

\textbullet 添加到Git仓库,分为两步

\qquad \textbullet 第一步,使用命令\textit{git add}, 可以反复多次使用,
	添加多个文件。

\qquad \textbullet 第二步,使用命令\textit{git commit}, 完成.\textit{-m}
参数提供提交说明
\section{时光穿梭}
\textbullet 随时掌握工作区的状态,使用\textit{git status}命令。

\textbullet 如果\textit{git status}告诉有文件被修改过, 用\textit{git diff}
可以查看修改内容。
\subsection{版本回退}

\textbullet 查看历史记录,\textit{git log},参数\textit{- -pretty=oneline}每次提交信息显示在同一行

\textbullet \textit{HEAD}是当前版本,上一版本为\textit{HEAD \^},前100版本\textit{HEAD~100},回到之前版本的命令为\textit{git reset - -hard HEAD \^},回到当前版本找到相应的commit id, 使用命令如\textit{git reset - -hard 363333}
\section{远程仓库}

\textbullet 在用户主目录下,看看有没有.ssh目录,如果有,看此目录下有无id\_rsa和id\_rsa.pub。如果有直接跳到下一步,否则,打开shell,创建SSH Key:

\textit{ssh-keygen -t rsa -C "youremail@example.com"}

注意,邮件地址为自己的邮件地址。这样.ssh目录下会有id\_rsa和id\_rsa.pub两个文件,这两个就是SHH Key密钥,id\_rsa是私钥,不能泄漏;id\_rsa.pub是公钥,可以告诉别人。

\textbullet 登录Github,打开“Account settings“,“SSH Keys“页面。点击“Add SSH Key“, 填上任意Title,在Key文本框里粘贴id\_rsa.pub文件中的内容。点击“Add Key“,就可以看到添加的Key。
\subsection{添加远程库}

\textbullet 登录Github,点击右上角“Create a new repo“。创建新仓库。在“Reposity name“ 填入仓库名。

\textbullet 从本地关联库

\textit{git remote add origin git@github.com:appsoner/learngit.git}

\textbullet 将本地库的所有内容推送到远程库上。

\textit{git push -u origin master}

由于远程库是空的,我们第一次推送master分支时,加上\textit{-u}参数,Git不但会把本地的master分支内容推送到远程新的master分支,还会把本地的master分支与远程的master分支关联起来,在以后的推送或者拉取时就可以简化命令。

\textbullet 本次本地提交后,只要有必要,就可以使用命令\textit{git push origin master}推送新修改。
\subsection{从远程仓库克隆}
\textit{git clone}

\textbullet 克隆远程仓库

\textit{git clone git@github.com:appsoner/gitskills.git}

\textbullet Git支持多种协议, 包括https, 但通过ssh支持原生git协议速度最快。
\subsection{创建于合并分支}

查看分支 \textit{git branch}

创建分支 \textit{git branch name}

切换分支 \textit{git checkout name}

创建~$+$~切换分支 \textit{git checkout -b name}

合并某分支到当前分支 \textit{git merge name}

删除分支 \textit{git branch -d name}
\subsection{解决冲突}
当Git无法自动合并分支时,就必须首先解决冲突。解决冲突后,在提交,合并完成。

用\textit{git log - -graph}命令可以看到分支合并图。如

\textit{git log - -graph - -pretty=oneline - -abbrev-commit	}
\end{document}

